\documentclass[pdf]{beamer}
\mode<presentation>{}
\usepackage{graphicx}
%\usepackage{lmodern}


%sources:

%http://www.vlab.msi.umn.edu/people/download/InternshipSummer2007/handler%20pres.pdf
%http://qubit-ulm.com/wp-content/uploads/2012/04/Lanczos_Algebra.pdf
%http://www.ams.org/bookstore/pspdf/mbk-76-prev.pdf

\usetheme{Warsaw}
\usecolortheme{dolphin}

\title{The Lanczos Algorithm}
\author{Shane Harding}

\begin{document}


\begin{frame}
\titlepage
\end{frame}

%%%%%%

\begin{frame}
\frametitle{Outline of talk}
\tableofcontents[]

\end{frame}


\section{Introduction}

\begin{frame}
\tableofcontents[currentsection]
\end{frame}

%%%%%%%%%%%%%%%%%	INTRO	%%%%%%%%%%%%%%%%%%%%%%%

\begin{frame}
\frametitle{Introduction}
\centering


\end{frame}

\begin{frame}
\centering
\frametitle{Brief History}
The Lanczos algorithm was developed by Cornelius Lanczos. He developed the method while working at Boeing in the 40s. It was then forgotten for a number of years due to the lack of computers meaning that it couldn't be properly utilised. When it was then ``rediscovered" it was modified multiple times to fix some instability issues.
\end{frame}



%%%%%%%%%%%%%	THE ALGORITHM	%%%%%%%%%%%%%%%%%%%%%



\begin{frame}
\tableofcontents[currentsection]
\end{frame}



\begin{frame}
The Lanczos Algorithm is used to solve large scale eigenvalue problems. So give a large $n \times n$ matrix $A$ we have:
\begin{exampleblock}{Eigenvalues and eigenvectors}
\[
A v_i = \lambda_i v_i
\]
\end{exampleblock}
Where the vectors $v_i$ are the eigenvectors and the scalars $\lambda_i$ are the eigenvalues.
\end{frame}


%%%%%%%%%%%%%%%%%%%	APPLICATIONS	%%%%%%%%%%%%%%%%%%%%



\begin{frame}
\tableofcontents[currentsection]
\end{frame}


\section{Applications}

\begin{frame}
\centering
Thank you!
\end{frame}

\end{document}
