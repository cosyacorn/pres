\documentclass[pdf]{beamer}
\mode<presentation>{}
\usepackage{graphicx}
\usepackage{amsmath}
\usepackage{amsfonts}
%\usepackage{lmodern}


%sources:

%http://www.vlab.msi.umn.edu/people/download/InternshipSummer2007/handler%20pres.pdf
%http://qubit-ulm.com/wp-content/uploads/2012/04/Lanczos_Algebra.pdf
%http://www.ams.org/bookstore/pspdf/mbk-76-prev.pdf

\usetheme{Warsaw}
\usecolortheme{dolphin}

\title{The Lanczos Algorithm}
\author{Shane Harding}

\begin{document}


\begin{frame}
\titlepage
\end{frame}

%%%%%%

\begin{frame}
\frametitle{Outline of talk}
\tableofcontents[]

\end{frame}

%%%%%%%%%%%%%%%%%	INTRO	%%%%%%%%%%%%%%%%%%%%%%%
\section{Introduction}

\begin{frame}
\tableofcontents[currentsection]
\end{frame}



\begin{frame}
\frametitle{Introduction}
\centering


\end{frame}

\begin{frame}
\centering
\frametitle{Brief History}
The Lanczos algorithm was developed by Cornelius Lanczos. He developed the method while working at Boeing in the 40s. It was then forgotten for a number of years due to the lack of computers meaning that it couldn't be properly utilised. When it was then ``rediscovered" it was modified multiple times to fix some instability issues.
\end{frame}



%%%%%%%%%%%%%	THE ALGORITHM	%%%%%%%%%%%%%%%%%%%%%


\section{The Algorithm}
\begin{frame}
\tableofcontents[currentsection]
\end{frame}



\begin{frame}
The Lanczos Algorithm is used to solve large scale eigenvalue problems. So give a large $n \times n$ matrix $A$ we have:
\begin{exampleblock}{Eigenvalues and eigenvectors}
\[
A v_i = \lambda_i v_i
\]
\end{exampleblock}
Where the vectors $v_i$ are the eigenvectors and the scalars $\lambda_i$ are the eigenvalues.
\end{frame}


\begin{frame}
\frametitle{Power Iteration}
\begin{itemize}

\item The Lanczos algorithm is an adaptation of the power method.
\item The power methods (or power iteration or Von Mises iteration) is an eigenvalue algorithm.
\item This algorithm will only produce one eigenvalue and may converge slowly.
\item It finds the eigenvalue with the greatest absolute value.

\end{itemize}
\end{frame}

\begin{frame}
\frametitle{Power Iteration}

The power method can be summarised as follows: If $x_0$ is some random vector and $x_{n+1}=A x_n$ then if we consider the limit of $n$ being large we find that $\frac{x_n}{||x_n||}$ approaches the normed eigenvector with the greatest eigen value. 
\begin{itemize}
\item Then if $A=U {diag} (\sigma_i) U'$ is the eigendeecomposition of $A$ then $A^n = U {diag}(\sigma^n_i) U'$.
\item As n gets big this will be dominated by the largest eigenvalue
\end{itemize}
\end{frame}



\begin{frame}
\frametitle{The Lanczos Algorithm}

We wish to solve the system:
\[Ax=b\]

where $A \in \mathbb{R}^{N \times N}$ and $b \in \mathbb{R}^N$. Let $x_0 \in \mathbb{R}^N$ be our initial guess at the solution, and $r_0=b-Ax_0$ be the residual of our guess.

We will calculate the tridiagonal, symmetric matrix $T_{mm} = V_m^* A V_m$

Let $v_1=\frac{r_0}{\beta_1}$ where $\beta_1=||r_0||$

\end{frame}


\begin{frame}[fragile]

\begin{verbatim}
1. set r_0 = b - Ax_0; beta_1 = abs(r_0); v_1 = r_0/beta_1;
2. for i = 1,...,m do
     r_i = A v_i - beta_i v_{i-1}
     alpha_i = (r_i,v_i)
     r_i = r_i - alpha_i v_i
     beta_{i+1} = abs(r_i)
     if(beta_{i+1} < tau)
        m = i
        break
     else
        v_{i+1} = r_i/beta_{i+1}
\end{verbatim}



\end{frame}

%%%%%%%%%%%%%%%%%%%	APPLICATIONS	%%%%%%%%%%%%%%%%%%%%


\section{Applications}

\begin{frame}
\tableofcontents[currentsection]
\end{frame}


\begin{frame}
\centering
Thank you!
\end{frame}

\end{document}
